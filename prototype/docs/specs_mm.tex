\documentclass[a4paper,12pt]{article}
\usepackage{fullpage,times}
\usepackage{verbatim}  % for comments
% \usepackage{euro}
% \EUR~10.10

%\input{Figs}
\input mm_macros.tex

\DeclareMathAlphabet{\mathbit}{OT1}{cmr}{bx}{it}
\def\vv#1{\ensuremath{\mathbit{\bf #1}}}


\begin{document}
\thispagestyle{empty}
\pagestyle{plain}
\centerline{\Large\bf Comments on the specs03.pdf, version of Sep.~15}
\bigskip

\centerline{\large\bf {\em Marek}}
\bigskip

\centerline{\sc \today}

\bigskip
\bigskip

{\em Note:} This draft is rather an illustration how to write the SMS
than an advanced SMS draft.
The purpose of this draft is to provide a background for discussion on how
to proceed.

\subsection*{Editorial}
Please ALWAYS place your name and the current date on ANY document.

\section{Symbolic model specification (SMS)}
The SMS documents all model entities (indexing structure, variables,
parameters, relations).
This draft is an incomplete SMS aimed at illustrating the SMS structure
and basic elements.

\subsection{The model purpose}
The model aims at supporting analysis of the relations between the decisions
and consequences of their implementation.
The latter is represented by the outcome variables.

\subsection{Indexing structure}
\subsubsection{Intro to indexing structure}
The SMS uses the Structured Modeling (SM) concepts; in particular the
compound entities.
For example, a compound variable $\vv{x}$ actually represents a set of variables
$x_{ij}, i \in I,\; j \in J$, where $i$ and $j$ stand for indices, and $I$ and
$J$ are the sets of values of the corresponding indices.
To illustrate this concept, let $\vv{x}$ be flows between $i$-th warehouse
and $j$-th store.
Then the set $I$ of warehouses can be defined as $I = \{city1, city2, \dots\}$.
Similarly, the set $J$ of shops can be defined as $J = \{loc1, loc2, \dots\}$.

Thus, the indexing structure is composed of:
\btlb
\item symbols of indices (typically a lower-case letter), and
\item symbols of sets (typically, the corresponding upper-case letter).
\etl
The examples of the index-sets below are for illustration only.
The actual members of these sets are defined by the model parameters.

\subsubsection{Indexing structure of the model}\label{sec:index}
The model uses the following indices and the corresponding sets:
\btlb
\item $t \in T$ technologies, $T = \{ {\rm OTL, BTL, PTL,} \dots \}$,
	$T_f \subset T$ technologies directly producing the final commodities
\item $y \in Y$ 5-year period id, $Y = \{ {\rm 2020, 2025, \dots 2050} \}$,
\item Using the convention for periods defined by the sequence (2020, \dots, 2050)
	causes complicated definitions of the corresponding relations (see below).
	Therefore I would consider to use another index for periods, say~$p \in P$,
	where $P = \{-\tau, -tau +1, \dots, 1, 2, \dots, 7\}$, where non-positive
	values correspond to historical (before the planning period) periods.
	This will allow easy definitions and use of historical and current new capacities.
	The correspondence between $Y$ and $P$ can be defined by a simple mapping.
	In the formulae below I use the $p \in P$ instead of $y \in Y}.\footnote{
	Let's discuss this issues.}
\item $v \in V$ vintage (construction time) period id,
	$V = H \cup \{ {\rm 2020, 2025, \dots 2050} \}$,
	where $H$ is the set of historical (previous) periods for which the
	corresponding new capacities are defined by the data,
\item $c \in C$ commodity, $C = \{ {\rm oil, gasoline, coal, crude-oil, \dots } \}$,
	$C_f \subset C$: final commodities,
\item \dots
\etl
{\em Notes for Jinyang}:
\btlb
\item We probably need to define subsets of~$C$, e.g., for dealing with final
	commodities.
\item I strongly recommend to refrain from using $v^y$ (or any other symbol defined
	by a letter with subscript/superscript) for an index.
	Therefore, I propose to use~$v$ for the vintage year index.
\etl

\subsection{Variables}
Although all variables are treated equally within the model,
we divide the set of all model variables into categories corresponding to the roles;
this helps for structuring the model presentation.

\subsubsection{Decision variables}
Decision-makers control the modeled system by decision (control) variables:
\btlb
\item $ncap_{tv}, t \in T, v \in V$: new production capacity of $t$-th
	technology, made available at the beginning of $p$-th period,
\item $act_{tvp}, t \in T, v \in V, p \in P$: activity level of $t$-th
	technology, using in period $p$ the new capacity provided in period~$v$.
\etl
{\em Editorial notes}:
\btlb
\item We use\footnote{JZ: it is up to you, which of these two notations
	you want to use. For the compactness I usually use the short one.}
	a short notation, i.e., $x_{ijk}$ instead of $x_{i,j,k}$.
\item Further on we skip the obvious explanations of the meaning of the
	indices.
\etl

\subsubsection{Outcome variables}
Outcome variables are used for evaluation of the consequences of implementation
of the decisions; therefore at least one of them is used as the optimization
objective.

In the model prototype only two outcomes (both used as criteria in
multiple-criteria model analysis) are defined:
\btlb
\item $cost$: the total cost of the system over the planning period, and
\item $CO2$: the total CO2 emission caused by the system.
\etl

\subsubsection{State variables}
The variables defining the state of the system:
\btlb
\item $cap_{tp}$: production capacity. {\em Note: this may not be needed,
	let's consider the relations defined in Section~\ref{sec:rel} and discuss.}
\item \dots
\etl

\subsubsection{Auxiliary variables}
All other variables used in the SMS:
\btlb
\item \dots
\etl

\subsection{Parameters}
The following model parameters are used in the model relations 
specified in in Section~\ref{sec:rel}:
\btlb
\item values of indices (members of sets) specified in Section~\ref{sec:index},
\item $\tau_t$: lifetime (number of periods) of the new capacity,
\item $d_{cp}, c \in C_f$: demand for final commodities defined by $C_f \subset C$
\item $a_{tvc}$: amount of product from the unit of the corresponding activity
\etl

\subsection{Relations}\label{sec:rel}
The values of the model variables conform to the following model relations.
\btlb
\item The sum of activities $act_{tvp}$ shall shall result in producing the required
	amounts of the final commodities:
	\be
	\sum_{t \in T_f} \sum_{v \in V_{tp}} a_{tvc} \cdot act_{tvp} \ge d_{cp} \quad
		c \in C_f, p \in P.
	\ee
	where $V_{tp} \subset V$ is defined by:
	\be
		V_{tp} = \{p - tau_t, p - tau_t + 1, \dots, p\}
	\ee
\item The levels of activities cannot exceed the corresponding capacities:
	\be
		act_{tvp} \le ncap_{tvp}, \quad t \in T, v \in V_p, p \in P.
	\ee
	Note: The values new capacities $ncap_{tvp}$ within the planning period
	($v \in P$) are defined by the decision variables.
	However, for the non-positive values of $V_{tp}$ (i.e., historical investments)
	are defined by the model parameters.
\etl




\newpage
\section*{Copy of comments of Sep.~1, 2021}
In order to assure consistency between the problem formulation and the
corresponding model, I would consider replacement of the obviously incorrect
inequality~(9) by:
\btla{88.}
\inum Equation defining the amounts of each of the fuels as a function of
the activities of the applied technologies:
	\be
	\sum_{i \in I} a_{ji} \cdot ACT_i^t = x_j^t, \quad j \in J,\; t \in T
	\ee
	where:
	\btlbs
	\item $j$ denotes fuel type, $J = \{gasoline, diesel\}$,
	\item $a_{ji}$ relates the amount of $j$-th fuel produced by the unit of
		$i$-th ACT,
	\item $x_j^t$ stands for the amount of $j$-th fuel produced jointly by all
		considered technologies at period~$t$.
	\etls
\inum Adding the supply-demand constraint, specification of which depends on the
	chosen definition of demand. Here we can consider one of the following two options:
	\btlas{88.}
	\inums If the demand is given for each fuel type, i.e., as $d_j^t$, then:
		\be\label{eq:dem1}
		x_j^t \ge d_j^t, \quad j \in J,\; t \in T.
		\ee
	\inums If the demand is given for a linear aggregation of $d_j^t$, e.g.,
		by coefficients $\alpha_j$ conforming to:
		\be
		0 \le \alpha_j \le 1,\; \forall j \in J; \quad \sum_{j \in J} \alpha_j = 1.
		\ee
		Thus, the demand is given for a {\em virtual} (i.e., not actually existing)
		fuel:
		\be
			d^t = \sum_{j \in J} \alpha_j \cdot d_j^t, \quad t \in T.
		\ee
		In such a case instead constraint~(\ref{eq:dem1}) one shall add constraint:
		\be
		\sum_{j \in J} \alpha_j \cdot x_j^t \ge d^t, \quad t \in T.
		\ee
	\etls
\etl

\end{document}
