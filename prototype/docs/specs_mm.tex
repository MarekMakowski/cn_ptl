\documentclass[a4paper,12pt]{article}
\usepackage{fullpage,times}
\usepackage{verbatim}  % for comments
% \usepackage{euro}
% \EUR~10.10

%\input{Figs}
\input mm_macros.tex

\DeclareMathAlphabet{\mathbit}{OT1}{cmr}{bx}{it}
\def\vv#1{\ensuremath{\mathbit{\bf #1}}}

\usepackage{amsmath}
\usepackage{lineno}
% \usepackage{showlabels}

\pagestyle{plain}
\thispagestyle{plain}


\begin{document}
\thispagestyle{empty}
\pagestyle{plain}
\centerline{\Large\bf Comments on the specs03.pdf, version of Sep.~15}
\bigskip

\centerline{\large\bf {\em Marek}}
\bigskip

\centerline{\sc \today}

\bigskip
\bigskip

\linenumbers

{\em Note:} This draft contains a still incomplete SMS.

\subsection*{Editorial}
Please ALWAYS place on ANY document: your name, the current date, page
numbers, and line numbers.

\section{Symbolic model specification (SMS)}
The SMS documents all model entities (indexing structure, variables,
parameters, relations).
This draft is an incomplete SMS aimed at illustrating the SMS structure
and basic elements.

\subsection{The model purpose}
The model aims at supporting analysis of the relations between the decisions
on the use-level of the technologies for producing liquid fuels, and the consequences
of implementation of such decisions.
The consequences are measured by values of the outcome variables.

\subsection{Indexing structure}
\subsubsection{Intro to indexing structure}
The SMS uses the Structured Modeling (SM) concepts; in particular the
compound entities.
For example, a compound variable $\vv{x}$ actually represents a set of variables
$x_{ij}, i \in I,\; j \in J$, where $i$ and $j$ stand for indices, and $I$ and
$J$ are the sets of values of the corresponding indices.
To illustrate this concept, let $\vv{x}$ be flows between $i$-th warehouse
and $j$-th store.
Then the set $I$ of warehouses can be defined as $I = \{city1, city2, \dots\}$.
Similarly, the set $J$ of shops can be defined as $J = \{loc1, loc2, \dots\}$.

Thus, the indexing structure is composed of:
\btlb
\item symbols of indices (typically a lower-case letter), and
\item symbols of sets (typically, the corresponding upper-case letter).
\etl
The examples of the index-sets below are for illustration only.
The actual members of these sets are defined by the model parameters.

\subsubsection{Indexing structure of the model}\label{sec:index}
The model uses the following indices and the corresponding sets:
\btlb
\item $t \in T$ technologies: $T = \{ {\rm OTL, BTL, PTL,} \dots \}$,
	$T_f \subset T$ technologies directly producing the final commodities
\item $p \in P$: the ids of the 5-year planing periods.
	Instead of using the calendar-year values, we use sequence of positive integer
	values, i.e., for the calendar years \{2020, 2025, \dots, 2050\} we index
	the planning periods by $P = \{1, 2, \dots, 7\}$.
	The correspondence between the planning periods and the calendar years
	can be defined (for reporting) by a simple mapping.
\item $v \in V$: vintage (period in which the capacity becomes available) period id.
	The duration (consecutive periods during which the capacity remains available)
	of each newly installed capacity is defined by the parameter~$\tau$;
	e.g, $\tau = 4$ defines the duration of the capacity equal to 4 periods,
	i.e., 20 years.
	Therefore, $V = H \cup P$ where $H$ is the set of historical (prior to the
	first planning period) periods in which the installed capacities can still
	be used during the planning periods.
	Thus, for $\tau$ equal to 4, $H = \{-2, -1, 0\}$, i.e., the capacities
	installed in any of the three periods before the first planning period
	are available to be used in the first planning period.
\item $c \in C$: commodity. E.g.,
	$C = \{ {\rm oil, gasoline, coal, crude-oil, \dots } \}$.
	Commodities belong to diverse subsets that correspond to their roles:
	\btlas{$\diamond$}
	\item $C_f \subset C$: final commodities (currently: gasoline and diesel-oil),
	\item $C_i \subset C$: commodities required as inputs for activities of
		technologies~$t \in T$,
	\item \dots  (more subsets shall be defined).
	\etls
\item \dots (more indices may be needed).
\etl
{\em Notes for Jinyang}:
\btlb
\item I strongly recommend to refrain from using $v^y$ (or any other symbol defined
	by a letter with subscript/superscript) for an index.
	Therefore, I propose to use~$v$ for the vintage period index.
\item We don't need in the SMS:
	$y \in Y$ 5-year period id, $Y = \{ {\rm 2020, 2025, \dots 2050} \}$.
\etl

\subsection{Variables}
Although all variables are treated equally within the model,
we divide the set of all model variables into categories corresponding to the roles;
this helps for structuring the model presentation.

\subsubsection{Decision variables}
Decision-makers control the modeled system by decision (control) variables:
\btlb
\item $cap_{tv}, t \in T, v \in P$: new production capacity of $t$-th
	technology, made available at the beginning of $v$-th period,\footnote{
	I don't think we need "new" capacities, because capacities are indexed
	by $v$, which means that each capacity was "new" in $v$-th period, and
	remains available in this period as well as in $\tau - 1$ following periods.}
\item $act_{tvp}, t \in T, v \in V, p \in P$: activity level of $t$-th
	technology, using in period $p$ the new capacity provided in period~$v$.
\etl
{\em Editorial notes}:
\btlb
\item We use\footnote{JZ: it is up to you, which of these two notations
	you want to use. For the compactness I usually use the short one.}
	a short notation, i.e., $x_{ijk}$ instead of $x_{i,j,k}$.
\item Further on, we skip the obvious explanations of the meaning of the indices.
\etl

\subsubsection{Outcome variables}
Outcome variables are used for evaluation of the consequences of implementation
of the decisions; therefore, at least one of them is used as the optimization
objective.

In the model prototype only two outcomes (both used as criteria in
multiple-criteria model analysis) are defined:
\btlb
\item $cost$: the total cost of the system over the planning period, and
\item $CO2$: the total CO2 emission caused by the production system.
\etl

\subsubsection{State variables}
The variables defining the state of the system:
\btlb
\item \dots (to be defined, if needed).
\etl

\subsubsection{Auxiliary variables}
All other variables used in the SMS:
\btlb
\item $actInp_{cp}$ input resources required by all technologies $t \in T$
\item \dots (all remaining variables, to be defined).
\etl

\subsection{Parameters}
The following model parameters are used in the model relations 
specified in in Section~\ref{sec:rel}:
\btlb
\item values of indices (members of sets) specified in Section~\ref{sec:index},
\item $\tau$: lifetime (number of periods) of the new capacity (equal for all
	technologies),
\item $d_{cp}, c \in C_f$: demand for final commodities defined by $C_f \subset C$
\item $a_{tvc}$: amount of output from the unit of the corresponding activity
\item $inp_{ctv}$: amount of input required by the unit of the corresponding activity
\item $ef_{tv}$: CO2 emission factor
\item $invC_{tv}$: unit investment cost of new capacity
\item $omcC_{tv}$: unit OMC of activity
\item $extP_{cp}$: unit price of external input resources
\item $utlF_{t}$: capacity utilization factor
\item \dots (the list is incomplete)
\etl

\subsection{Relations}\label{sec:rel}
The values of the model variables conform to the following model relations.
\btlb
\item The sum of activities $act_{tvp}$ shall result in producing the required
	amounts of the final commodities:
	\be
	\sum_{t \in T_f} \sum_{v \in V_p} a_{tvc} \cdot act_{tvp} \ge d_{cp} \quad
		c \in C_f, p \in P.
	\ee
	where $V_p \subset V$ is defined by:
	\be \label{eq:vp}
		V_p = \{p - \tau + 1, p - \tau + 2, \dots, p\}.
	\ee
	The parameter $\tau > 1$ defines the number of following periods during which the
	capacity installed in $v$-th period remains available for the corresponding
	activity. In the prototype implementation, the value of~$\tau$ is equal
	for all technologies.\footnote{
	Note that for $\tau = 4$ (the capacity life-time of 20 years) the
	$V_p$ defined by~\eqref{eq:vp} is equal to
	$V_p = \{p - 3, p - 2, p - 1, p\}$.}
\item The levels of activities cannot exceed the corresponding available
	capacities:
	\be
		act_{tvp} \le utlF_t \cdot cap_{tv}, \quad t \in T, v \in V_p, p \in P.
	\ee
	Note: The values new capacities $cap_{tv}$ within the planning period
	($v \in P$) are defined by the decision variables.
	However, for the non-positive values of $V_{p}$ (i.e., historical investments)
	the $cap_{tv}$ values are defined by the model parameters.
\item Each activity requires the corresponding input resources.
	The following relation defines the inputs of the corresponding commodities
	required as inputs by all technologies:
	\be
		actInp_{cp} = \sum_{t \in T} \sum_{v \in V_p} inp_{ctv} \cdot act_{tvp},
		\quad  c \in C_i,\; p \in P
	\ee
\item Note: relations describing the up-stream technologies need to be added here
\item The total CO2 emission caused by the activities is defined by:
	\be
		CO2 = \sum_{p \in P} \sum_{t \in T} \sum_{v \in V_p} ef_{tv} \cdot act_{tvp}
	\ee
	Note: emissions from up-stream activities shall be added after these will be defined.
\item Investment costs are defined by:
	\be
		invCost = \sum_{t \in T} \sum_{v \in P} invC_{tv} \cdot cap_{tv}
	\ee
\item Operations and maintenance costs are defined by:
	\be
		omc = \sum_{t \in T} \sum_{p \in P} \sum_{v \in V_p} omcC_{tv} \cdot act_{tvp}
	\ee
\item Costs of externally provided inputs are defined by:
	\be
		extCost = \sum_{c \in C_i} \sum_{p \in P} extP_{cp} \cdot actInp_{cp}
	\ee
\item Total cost is defined by:
	\be
		cost = invCost + omc + extCost
	\ee
\etl




\newpage
\section*{Copy of comments of Sep.~1, 2021}
In order to assure consistency between the problem formulation and the
corresponding model, I would consider replacement of the obviously incorrect
inequality~(9) by:
\btla{88.}
\inum Equation defining the amounts of each of the fuels as a function of
the activities of the applied technologies:
	\be
	\sum_{i \in I} a_{ji} \cdot ACT_i^t = x_j^t, \quad j \in J,\; t \in T
	\ee
	where:
	\btlbs
	\item $j$ denotes fuel type, $J = \{gasoline, diesel\}$,
	\item $a_{ji}$ relates the amount of $j$-th fuel produced by the unit of
		$i$-th ACT,
	\item $x_j^t$ stands for the amount of $j$-th fuel produced jointly by all
		considered technologies at period~$t$.
	\etls
\inum Adding the supply-demand constraint, specification of which depends on the
	chosen definition of demand. Here we can consider one of the following two options:
	\btlas{88.}
	\inums If the demand is given for each fuel type, i.e., as $d_j^t$, then:
		\be\label{eq:dem1}
		x_j^t \ge d_j^t, \quad j \in J,\; t \in T.
		\ee
	\inums If the demand is given for a linear aggregation of $d_j^t$, e.g.,
		by coefficients $\alpha_j$ conforming to:
		\be
		0 \le \alpha_j \le 1,\; \forall j \in J; \quad \sum_{j \in J} \alpha_j = 1.
		\ee
		Thus, the demand is given for a {\em virtual} (i.e., not actually existing)
		fuel:
		\be
			d^t = \sum_{j \in J} \alpha_j \cdot d_j^t, \quad t \in T.
		\ee
		In such a case instead constraint~(\ref{eq:dem1}) one shall add constraint:
		\be
		\sum_{j \in J} \alpha_j \cdot x_j^t \ge d^t, \quad t \in T.
		\ee
	\etls
\etl

\end{document}
